\section{Introduction}
\label{sec:introduction}

  % With the collapse of Denard scaling and Moore's law, architects are transitioning towards multi-core systems. This allows multiple cores on a single die, connected by high bandwidth interconnects. 
  % However, these interconnects still cause most applications to be memory bound. Moreover, interprocess communication in latency-critical applications suffers from the overhead of communicating via the operating system mechanisms. 

  % With the growth of edge computing and embedded systems, the need for real-time low-latency communication is growing. Autonomous vehicles, drones, robotics, and augmented/virtual realtity applications require low-latency communications to synchronize multi-core systems.

  % Current solutions either incorporate elaborate hardware interconnect fabrics, such as the AMD Infinity Fabric, or rely on software-based solutions such as shared memory. We focus on shared memory solutions as they are the principle means of communication for application programmers.
  % However, shared memory often requires the use of locks, the cache hierarchy, and suffers nondeterministic latency from TLB flushes and cache pollution.

  % In this paper, we present RVipc, a low-latency inter-core communication protocol fo real-time systems. RVipc is designed for trusted environments, such as embedded systems. 

  Multi-core systems are becoming increasingly fundamental to modern computing, from servers down to embedded systems and edge devices. This paradigm shift is driven by the need for higher compute power, as traditional scaling methods have plateued. 

  Multi-core systems incorporate multiple cores on a single die, often tied together by high-bandwidth interconnects. For example, AMD utilizes Infinity Fabric, a proprietary interconnect technology, to connect multiple chips and cores.

  There are several challenges with these forms of interconnects. First, they are usually designed for high bandwidth, and often cause nondeterministic latencies. This could be due to the routing algorithms or contention. More importantly, these interconnects are software-transparent, meaning application programmers cannot directly control how traffic flows across them. 
  This limitation motivates the use of shared memory as the primary means of communication.

  Shared memory suffers from nondeterministic latencies due to the cache hierarchy, TLB flushes, and cache pollution.

  This is problematic for latency-critical applications such as autonomous vehicles, drones, robotics, and augmented/virtual reality applications. The success and safety of these applications directly depends on low-latency synchronization and communication between cores. A failure to do so significantly lowers the quality and feel of these products. 

  This motivates the need for a low-latency inter-core commmunication protocol. In this paper, we present RVipc, a low-latency inter-core communication protocol designed for trusted environments. Our contributions are as follows:

  \begin{enumerate}
    \item Deterministic latencies in communication between cores
    \item Low-latency communication between cores 
    \item Software tooling to automatically communicate between cores in the most efficient manner
  \end{enumerate}
